%Template für Kapitel
%Da als Dokumentenklasse Report gewählt wurde und für den Report 'Chapter' die höchste Gliederungsstufe darstellt, beginnt jedes Kapitel mit Chapter.
\chapter{Einleitung}
\section{Problemstellung}
Jeder Gitarrist strebt an, aus seinem Instrument das Maximum an Klang heraus zu holen. Dazu werden verschiedene Ansätze verfolgt. Dies können außergewöhnliche Variationen der Bauweise oder die Wahl der Tonabnehmer sein. Der gängigste Ansatz ist jedoch das gezielte Verzerren des Gitarrensignals durch Effekt-Geräte, welche zwischen Gitarre und Verstärker, und per Fußschalter an und aus geschaltet werden. Diese erzeugen zumeist hervorragende Klänge, jedoch haben Sie entschiedene Nachteile. Möchte man mehr als einen Effekt gleichzeitig umschalten wird dieser Versuch oft zur motorischen Herausforderung, ist das Lied schnell, kommt es vor, dass der Effekt erst nach Beginn des entsprechenden Takts erklingt. Des Weiteren können einige Effekt-Geräte über zusätzliche Potentiometer konfiguriert werden. Ein Ändern dieser Feineinstellungen während eines Liedes ist nahezu unmöglich.

Abhilfe können hier Verstärker schaffen, welche über ihren eigentlichen Zweck hinaus gehen, und verschieden Kanäle für unterschiedliche starke Verzerrungen besitzen, oder aber verschiedene Effekte an Bord haben. Das bloße Auslagern der dieser Funktionalitäten in den Verstärker bietet hier offensichtlich noch keinen Mehrwert. Erst durch das Speichern einer bestimmten Verstärkerkonfiguration, sowie das anschließende Laden per Tritt auf den Fußschalter, lassen sich mehrere Effekte gleichzeitig anschalten.

Dies geschieht bei meinem Grandmeister 36 der Firma Hughes \& Kettner, in mir noch unbekannter Form, über das \index{MIDI}{MIDI-Protokoll}.
Einen solchen Fußschalter besitze ich zwar, es erscheint mir jedoch so, dass dieser das Potential der MIDI-Schnittstelle nicht im vollen Umfang ausschöpft. Es wäre denkbar eine auf dem Einplatinencomputer \index{Raspberry Pi}{Raspberry Pi} basierende Schaltung, so wie ein entsprechendes Programm zu entwerfen. Damit könnte die Funktion eines solchen Fußschalters umgesetzt und beliebig erweitert werden.

\section{Motivation}

Meine Motivation ist eine Eigenbau-Lösung für die genannte Problemstellung zu finden. Dies bietet gleich mehrere Möglichkeiten. Zum einen ist man nicht auf den Entwurf der Hersteller angewiesen, sondern kann sich die Hardware nach funktionalen aber auch ästhetischen Gesichtspunkten gestalten wie man möchte. Zum anderen hat man die gleichen Freiheiten in der Software, und kann sich vielleicht sogar Funktionen schreiben, die noch besser auf die eigenen Bedürfnisse passen. Außerdem glaube ich, dass es sich hierbei um ein Projekt handelt, an dem ich meine theoretischen Kenntnisse aus dem Studium praktisch vertiefen und so noch einiges Lernen kann.

Ein weiter didaktischer Aspekt ist das Analysieren, Verstehen und Anwenden eins unbekannten Protokolls, in diesem Falle des MIDI-Protokoll. Dieses wird in der Musik recht häufig verwendet. Elektrische-Schlagzeuge, Drumcomputer, verschiedene Keyboards, aber inzwischen auch einige E-Gitarren benutzen das Protokoll um eine noch breitere Auswahl an Klangbildern zu erzeugen. Das Aneignen dieser Technologie erscheint daher sinnvoll. 

\section{Projektziele}

	
	
	Grundlegendes Sachziel des Projekt ist es, eine auf einem Raspberry Pi basierte Schaltung, so wie ein entsprechendes Programm zu erstellen, welche in Kombination in der Lage sind MIDI-Befehle so an den Verstärker zusenden, so dass dieser damit Konfigurationen speichern und Laden kann. Ein ästhetischer Anspruch ist vorhanden, jedoch wird dieser innerhalb des Gesamtziels weniger schwer gewichtet. Eine genaue Definition der Soll- und Kann-Ziele wird in der Analysephase gefunden. 


 
	


