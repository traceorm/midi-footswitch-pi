%Template für Kapitel
%Da als Dokumentenklasse Report gewählt wurde und für den Report 'Chapter' die höchste Gliederungsstufe darstellt, beginnt jedes Kapitel mit Chapter.
\chapter{Projektziele}

\begin{flushleft}

Im folgenden werden die Zeile des Projekts formuliert. Sollte eins dieser Ziele nicht erfüllt werden, ist das Projekt als gescheitert anzusehen. Sollte dies geschehen muss in die Analysephase zurückgekehrt werden um zu evaluieren, ob das Projekt unter anderen Zielsetzungen noch erfolgreich sein kann, oder ob eine Weiterführung nicht statt findet.

\end{flushleft}

\subsection{Sachziel}
\begin{flushleft}

Grundlegendes Sachziel des Projekt ist es, eine Microcontroller basierte Schaltung, so wie ein entsprechendes Grundprogramm zu erstellen, welche in Kombination in der Lage sind MIDI-Befehle so an den Verstärker zusenden, so dass dieser damit Konfigurationen speichern und Laden kann. Eine genaue Definition der Soll- und Kann-Ziele wird in der Analysephase gefunden. 
\end{flushleft}
\subsection{Kostenziel}
\begin{flushleft}

Damit sich das Projekt auch wirtschaftlich einigermaßen lohnt, sollten zumindest die Materialkosten unter dem Preis eines neu angeschafften Fußschalters der Firma Hughes \& Kettner bleiben.  

\end{flushleft}


\subsection{Terminziel}

\begin{flushleft}

Da es sich hierbei um ein Embedded-Projekt bei Professor Scharschmidt (? wtf wird der geschrieben) handelt, muss es auch bis zur Prüfungsphase des Sommersemesters 2016 durchgeführt worden sein, da andernfalls eine Prüfung nicht stattfinden kann. Terminziel ist daher der 08.07.2016, da zu diesem Zeitpunkt die Vorlesungen des Sommersemesters enden.

\end{flushleft}


\section{Platzhalter für alles was noch zur Defintion fehlt}

\section{Meilensteine}
%da es sich um ein Studienprojekt handelt, das neben dem Studium, keine genauen Zeitvorgaben, jedoch Setzung von Meilensteinen, um zu gewährleisten das irgendwas von wegen in einander greifend, aber abhängig und so

\subsection{Abschluss Analyse}
\subsection{Abschluss Planung}
\subsection{Abschluss Durchführung}
\subsection{Abschluss Validierung}
\subsection{Abschluss Dokumentation}
