%%Benötigte Pakete

%utf8encoding für Umlaute und Ähnliches
\usepackage[utf8]{inputenc}  
\usepackage[english,ngerman]{babel}

%Paket zum Setzen der Ränder
\usepackage[left=2cm,right=2cm,top=2cm,bottom=2cm]{geometry}




%wird laut Braun benötigt, für ordentliche PDF-Ausgabe, macht bis jetzt keinen Unterschied.
%\usepackage[T1]{inputenc}


%Paket für Bilder
\usepackage{graphicx}
%konfiguriere den Pfad wo Latex nach Bildern suchen soll. Mehrere Pfade in weiteren Klammern möglich!
\graphicspath{ {bilder/} }

\usepackage{float}




%Paket für SI-Einheiten, weil Einheiten sind dein Freund
\usepackage{siunitx}

%paket zum Anzeigen von Quellcode
\usepackage{listings}

%paket zum Einfügen von Hyperlinks
\usepackage{hyperref}

%paket zur Manipulation von Spaltengrößen, mit Textausrichtung, da der tabular Parameter p{} die Parameter l,c,r ausschließt.

\usepackage{tabularx}
\newcolumntype{L}[1]{>{\raggedright\arraybackslash}p{#1}} % linksbündig mit Breitenangabe
\newcolumntype{C}[1]{>{\centering\arraybackslash}p{#1}} % zentriert mit Breitenangabe
\newcolumntype{R}[1]{>{\raggedleft\arraybackslash}p{#1}} % rechtsbündig mit Breitenangabe

%auf a4 nehmt ihr immer 11pt ok?



%%Einfügen des Pakets makeidx und erstellen des Index
%ACHTUNG KONFLIKT GEFAHR mit dem build Ordner. 
\usepackage{makeidx}
\makeindex