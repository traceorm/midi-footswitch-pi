%Template für Kapitel
%Da als Dokumentenklasse Report gewählt wurde und für den Report 'Chapter' die höchste Gliederungsstufe darstellt, beginnt jedes Kapitel mit Chapter.
\chapter{Einleitung}
Jeder Gitarrist strebt an, aus seinem Instrument das Maximum an Klang heraus zu holen. Dazu werden verschiedene Ansätze verfolgt. Dies können außergewöhnliche Variationen der Bauweise oder die Wahl der Tonabnehmer sein. Der gängigste Ansatz ist jedoch das gezielte Verzerren des Gitarrensignals durch Effekt-Geräte, welche zwischen Gitarre und Verstärker, und per Fußschalter an und aus geschaltet werden. Diese erzeugen zumeist hervorragende Klänge, jedoch haben Sie entschiedene Nachteile. Möchte man mehr als einen Effekt gleichzeitig umschalten wird dieser Versuch oft zur motorischen Herausforderung, ist das Lied schnell, kommt es vor, dass der Effekt erst nach Beginn des entsprechenden Takts erklingt. Des Weiteren können einige Effekt-Geräte über zusätzliche Potentiometer konfiguriert werden. Ein Ändern dieser Feineinstellungen während eines Liedes ist nahezu unmöglich.

Abhilfe können hier Verstärker schaffen, welche über ihren eigentlichen Zweck hinaus gehen, und verschieden Kanäle für unterschiedliche starke Verzerrungen besitzen, oder aber verschiedene Effekte an Bord haben. Das bloße Auslagern der Effekte in den Verstärker bietet hier offensichtlich noch keinen Mehrwert. Erst durch das Speichern einer bestimmten Verstärkerkonfiguration, sowie das anschließende Laden dieser auf Knopfdruck, lassen sich mehrere Effekte gleichzeitig anschalten.

Dies geschieht bei meinem Grandmeister 36 der Firma Hughes & Kettner über das MIDI-Protokoll \index{MIDI}. Einen solchen Fußschalter besitze zwar,  es erscheint jedoch so, dass dieser das Potential der Schnittstelle nicht im vollen Umfang ausschöpft. Es wäre denkbar eine auf dem RaspberryPi basierende Schaltung, so wie ein entsprechendes Programm zu entwerfen, die die Funktion eines solchen Fußschalters umsetzt und beliebig erweitert. 
