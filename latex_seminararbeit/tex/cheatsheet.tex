%%cheatsheet für latex


\subsection{Leerzeichen}
Manche Befehle verschlucken Leerzeichen:
\\
\LaTeX z.B. \\
\LaTeX \ ein Slash behebt dies.

\subsection{Bindestriche}

-- und - und ----- sind unterschiedlich

\section{Formatierung}

\subsection{Chars}
umlaute siehe Präambel, ü, ö, ß
Sonderzeichen mit Backslash escapen, oder eben nicht, einfach ausprobieren

@ \% \_ \{ \}

\subsection{logische und physische Formatierung}


%physische Formatierung
\textbf{Fetter Text}
\\
\bfseries
viiiiel dicker text am stück
\normalfont
\\und wieder normal\\
\textit{Kursiv}
\\
\textsc{Kapital}
\\
%logische Formatierung IST DEIN FREUND! manipulation später durch präambel
\emph{Betoner Text}
%lokale formatierung
\textbf{Fetter Text}
\textit{italic}
\textsl{slanted}
\textsc{Kapital}
%unäre Formatieung
\itshape
sdöldfjkkglkfjgösdflfsdldsas
\normalfont

\subsection{Schriftgrößen}
Lassen wir eigentlich so, bzw machen das wenn dan  über die Präambel.

WENN man das dann doch mal macht:

\tiny
Hier steht kleiner Text, der als Schalter Aktiviert wurde.
\Huge
Hier knallts ordentlich laut
\normalsize
und hier ist wieder alles normal

\section{Umgebung}

Mit begin und end

Innerhalb der Umgebung gelten neue Regeln
\begin{enumerate}
	\item
	test
	\item
	test2
\end{enumerate}

\begin{itemize}
	\item Punkte
	\item ganz
	\item Viele
\end{itemize}

\section{"Blöcke"}
Tipp vom Prof:

"Latex denkt in Blöcken, also Rechtecke, alles wird in Blöcken gesetzt"

Dinge explizit zusammen fassen mit \{ \}

Was macht denn wohl {\bf dieser Ausdruck}?

Das ist Tatsächlich sehr cool

\section{Quelltext Strukturierung}
\begin{itemize}
	\item Präambel
	\subitem Beschreibung
	\subitem Packages
	\item $\backslash$input\{file\} zum "weg"sortieren
\end{itemize}

\section{Variabeln}
\subsection{Längen}
Auch Variablen werden in \LaTeX \ mit  Backslash $\backslash$ angeführt

Beispielsweise $\backslash$setlength\{\}
%\setlength

bei $\backslash$setlength\{\} immer relative angaben

Längenangaben können mit $\backslash$the angezeigt werden:
\begin{enumerate}
	\item
	$\backslash$textheight : \the \textheight
	\item
	$\backslash$textwidth : \the \textwidth
	\item
	$\backslash$paperheight : \the \paperheight
	\item
	$\backslash$paperwidth : \the \paperwidth
\end{enumerate}

\subsection{Layout}
Überblick mit $\backslash$layout\{\}:




%\layout{}

\subsection{Manipualtion von Variabeln}
Überblick mit $\backslash$addtolength\{$\backslash$parindent\}:

\addtolength{\parindent}{1pt}
"In den letzten Jahren haben sich die Teilnehmerzahlen an unserer Konferenz stetig verdoppelt. Noch
einmal mache ich das nicht mehr mit", erklärt Andreas Hofmann, Haupt-Organisator der Konferenz und
Lehrer an der Waldschule Hatten, zum Auftakt der Veranstaltung. Der Ton etwas ironisch, und schon
zwei Wochen nach der Konferenz berichtet er von den Planungen für das nächste Jahr. Die
Herausforderung für eine Erweiterung der Veranstaltung sei angenommen, ein Ende sicherlich nicht in
Sicht.

\addtolength{\parindent}{1mm}
"In den letzten Jahren haben sich die Teilnehmerzahlen an unserer Konferenz stetig verdoppelt. Noch
einmal mache ich das nicht mehr mit", erklärt Andreas Hofmann, Haupt-Organisator der Konferenz und
Lehrer an der Waldschule Hatten, zum Auftakt der Veranstaltung. Der Ton etwas ironisch, und schon
zwei Wochen nach der Konferenz berichtet er von den Planungen für das nächste Jahr. Die
Herausforderung für eine Erweiterung der Veranstaltung sei angenommen, ein Ende sicherlich nicht in
Sicht.

\addtolength{\parindent}{1cm}
"In den letzten Jahren haben sich die Teilnehmerzahlen an unserer Konferenz stetig verdoppelt. Noch
einmal mache ich das nicht mehr mit", erklärt Andreas Hofmann, Haupt-Organisator der Konferenz und
Lehrer an der Waldschule Hatten, zum Auftakt der Veranstaltung. Der Ton etwas ironisch, und schon
zwei Wochen nach der Konferenz berichtet er von den Planungen für das nächste Jahr. Die
Herausforderung für eine Erweiterung der Veranstaltung sei angenommen, ein Ende sicherlich nicht in
Sicht.

\addtolength{\parindent}{1in}
\begin{flushleft}
	
	
	"In den letzten Jahren haben sich die Teilnehmerzahlen an unserer Konferenz stetig verdoppelt. Noch
	einmal mache ich das nicht mehr mit", erklärt Andreas Hofmann, Haupt-Organisator der Konferenz und
	Lehrer an der Waldschule Hatten, zum Auftakt der Veranstaltung. Der Ton etwas ironisch, und schon
	zwei Wochen nach der Konferenz berichtet er von den Planungen für das nächste Jahr. Die
	Herausforderung für eine Erweiterung der Veranstaltung sei angenommen, ein Ende sicherlich nicht in
	Sicht.
\end{flushleft}
\subsection{Eigene Längen}
%BEnutzen für alle wiederkehrenden Größen
%siehe tex/befehle.tex
%\newlength{\imgWidth}
%\setlength{\imgWidth}{\textwidth}

\subsection{Befehle}

siehe befehle.tex

\subsection{Zähler}
\setlength{\parindent}{3ex}


Es gibt einige Zähler, also quasi alles ist zählbar. Diese Können mit einigen Befehlen manipuliert werden. 


Eigene Counter können definiert werden.
\newcounter{eigenerCounter}
Und gesetzt werden. 
\setcounter{eigenerCounter}{2}
Counter können in ihrem Aussehen manipuliert werden:
\Roman{eigenerCounter}

\begin{enumerate}
	
	
	\renewcommand{\theenumi}{\alph{enumi}}
	\item test
	
\end{enumerate}

\begin{lstlisting}
\setcounter{page}{26}
\stepcounter{page}
\newCounter{eigenerCounter}
\setcounter{eigenerCounter}{10}
\end{lstlisting}

\section{label und Ref}

Dinge können in Latex gelabelt und später referenziert werden: labelkonvention asd:fgdh
\begin{lstlisting}
\label{}
\ref{}
\end{lstlisting}



